% Created 2019-11-10 Sun 17:22
% Intended LaTeX compiler: pdflatex
\documentclass[11pt]{article}
\usepackage[utf8]{inputenc}
\usepackage[T1]{fontenc}
\usepackage{graphicx}
\usepackage{grffile}
\usepackage{longtable}
\usepackage{wrapfig}
\usepackage{rotating}
\usepackage[normalem]{ulem}
\usepackage{amsmath}
\usepackage{textcomp}
\usepackage{amssymb}
\usepackage{capt-of}
\usepackage{hyperref}
\author{Tianchang WANG}
\date{\today}
\title{}
\hypersetup{
 pdfauthor={Tianchang WANG},
 pdftitle={},
 pdfkeywords={},
 pdfsubject={},
 pdfcreator={Emacs 26.3 (Org mode 9.1.9)}, 
 pdflang={English}}
\begin{document}

\tableofcontents

\section{The Church-Turing Thesis}
\label{sec:orgbbf0803}
\subsection{Turing machines}
\label{sec:org84f11b0}
\subsubsection{Introduction}
\label{sec:org5213146}
TM is a \textbf{much more powerful model}, it can do everything that a general computer can do. 

Nonetheless, even a Turing machine cannot solve certain problems. These problems are beyond the theoretical limit of computation.

\begin{itemize}
\item Uses an \textbf{infinite tape} as its unlimited memory. It has a tape head that can read and write symbols and move around on the tape.
\end{itemize}

Initially the tape contains only the \textbf{input string}, and is blank everywhere else. 

If the machine needs to store some information, it may \textbf{write information} on the tape. 

To read the information \emph{it has written}, the machine can move its head back over it. 

The machine continues computing until it decides to produce an output. The outputs \emph{accept} and \emph{reject} are obtained by entering designated accepting and rejecting states. 
\subsubsection{Differences between FA \& TM}
\label{sec:orgc769ac6}
\begin{enumerate}
\item A tm can both read and write the tape.
\item The read-write head can move both to the left and to the right.
\item The tape is \textbf{infinite}
\item The special states for rejecting and accepting take dffect immediately.
\end{enumerate}

\subsubsection{A example: TM to test membership of \{w\#w | w \(\in\) \{0,1\}\(^{\text{*}}\)\}}
\label{sec:org2f184da}

The langauge B means a string comprises two identical strings separated by a \# symbol.

Imagine you are on a mile long road with inputs on the ground. How to determine the input is in B? \textbf{Zig-zag} to the correspinding palces on the two sides of the \# and determine whether they match, is a obvious strategy.

And we design a machine to work in that way. 
M\(_{\text{1}}\) = "On input string \emph{w}
\begin{enumerate}
\item Zig-zag across the tape to correspinding positions on either side of the \emph{\#} symbol to check whether these positions contain the same symbol. 
If they do not, or if no \emph{\#} is found, \texttt{reject}. Cross off symbols as they are checked to keep track of which symbols correspond.
\item When all symbols to the left of the \texttt{\#} have been crossed off, check for any remaining symbols to the right of the \texttt{\#}, if any symbols remain, \emph{reject}; otherwise, \emph{accept}.
\end{enumerate}
"







\subsubsection{Formal Definition of a Turing Machine}
\label{sec:org4babc0c}

The heart: transition function \(\sigma\)
For a Turing machine, \(\sigma\) takes the form: Q X \(\gamma\) -> Q X \(\gamma\) X \{L,R\}. That is, when the machine is in a certain state q and the head is over a tape square containing a symbol a, and if \(\sigma\)(q,a) = (r,b,L)
-> The machine writes the symbol b replacing the a, and goes to state r. The third componnent is either L or R and indicates whether the head moves to the ledt or right after writing. In this case, the L indicates a move to the left.

A Turing machine is a 7-tuple, (Q, Σ, Γ, δ, q 0 , q accept , q reject ), where Q, Σ, Γ are all finite sets and
\begin{enumerate}
\item Q is the set of states,
\item Σ is the input alphabet not containing the blank symbol
\item Γ is the tape alphabet, where ␣ ∈ Γ and Σ ⊆ Γ,
\item δ: Q × Γ−→Q × Γ × \{L, R\} is the transition function,
\item q 0 ∈ Q is the start state,
\item q accept ∈ Q is the accept state, and
\item q reject ∈ Q is the reject state, where q reject ≠ q accept .
\end{enumerate}
\subsubsection{More on the definition:}
\label{sec:org3266596}
\capitalsigma 
\end{document}
